\documentclass{article}
\usepackage[utf8]{inputenc}
\usepackage{amsfonts}
\usepackage{amssymb}
\usepackage{amsmath}
\newtheorem{defi}{Définition}
\newtheorem{form}{Formules}
\title{(4) Nombres Complexes}
\author{Pim-Simon Hauguel}
\date{13-10-2020 (Trop tard)}

\begin{document}


\maketitle

\tableofcontents

\section{Excuses}
Bon déjà, première chose : Je suis désolé de vous avoir laché durant ces trop longs 30 jours qui nous sépare de mon dernier paper. La prochaine fois, il faut me taper sur les doigts ! Bref, je vais essayer de reprendre le rythme que je m'étais imposé au début et arreter de faire du hors programme. Commencons !

\section{Qu'est-ce qu'un nombre premier ?}
\begin{defi}[Définition et écriture]
  L'ensemble des nombres complexes est noté $\mathbb{C}$, on dit que
  $\mathbb{R} \subset \mathbb{C}$.
  L'unité dite imaginaire, est noté $i$ et est définie comme : $i = \sqrt{-1}$, donc que $i^{2} = -1$.
  La forme $z = a + ib$ est appelée \textit{notation cartésienne} de z (z est complexe).
  Avec $a$ la partie réelle ($Re(z)$), et $b$ la partie imaginaire ($Im(z)$). ($a,b \in \mathbb{R}^{2}$)
\end{defi}
\subsection{Ce que l'on en déduit}
Avec ces informations, certaines informations peuvent être remarquées :\\
\begin{itemize}
  \item $\forall z \in \mathbb{C},\exists! (a,b) \in \mathbb{R}^{2}(z = a + ib)$\\
    Pour le dire d'une manière plus accesible, un nombre complexe $z$ a une unique écriture cartésienne.
  \item $i\mathbb{R}$ est l'ensemble des imaginaires purs. Des exemples d'imaginaires purs ?
    \begin{itemize}
      \item $0 + 7i = 7i$
      \item $0 + i = i$
      \item $i^{3} = i^{2} * i = -1 * i = -i)$
    \end{itemize}
  \item Il existe aussi des réels purs, le principe est le même, simplement c'est la partie imaginaire qui est nulle.
\end{itemize}
\subsubsection{Le conjugué}
\begin{form}[Formule stricte]
  Le conjugué d'un nombre complexe $z$ est défini ainsi :

\centering{$conjugue(z) = \overline{z} = Re(z) - Im(z)$}
\end{form}
Pour résumé, le conjugué d'un nombre complexe est simplement ce même nombre, mais en opposant sa partie imaginaire.\\
On peut donc en déduire cela : $\overline{\overline{z}} = z$
\subsubsection{Minis Exos 1}
En connaissant ces informations, veuillez généraliser ces formules (trouver le $?$):
\begin{itemize}
  \item $\forall z \in \mathbb{C}(z*\overline{z} = ?)$
  \item $\forall z \in \mathbb{C}(z + \overline{z} = ? \in ?)$
  \item $\forall z \in \mathbb{C}(z \in i\mathbb{R} \Leftrightarrow \overline{z} = ?$
\end{itemize}
(bon ils étaient triviaux, mais bon...)

\section{Représentation des nombres complexes}
On peut exprimer les nombres complexes comme des points dans un plan.\\
Posons $z = a + ib$, $a$ va représenter l'abscisse, $b$ l'ordonnée. (Il est très facile de se visualiser cela dans le plan réel, cependant, nous sommes bien ici dans le plan complexe)\\
Par exemple, le nombre $8 + 7i$ sera de coordonnées $(8,7)$\\$-4 + 0.5i$ en $(-4,0.5)$\\

Posons $M$, un point du plan complexe de centre O,  de coordonnées $(a,b)$, quel sera sa longueur ? Il est évident que l'on répondra formule de pythagore, avec du coup $OM = \sqrt{a^{2} + b^{2}}$. Et bien je suis d'accord ! Mais avec les nombres complexes, on appelera cette longueur le module de $a + ib$.\\
\begin{form}{Module}
$\centering{\forall z \in \mathbb{C}(|z| = \sqrt{z\overline{z}} = \sqrt{a^{2} + b^{2}})~with~z~=~a~+~ib}$
\end{form}

\subsection{Les Vecteurs}
Les grandeurs dites scalaires, sont caractérisées par un seul nombre, a contrario des grandeurs dites vectorielles qui sont caractérisées par $n$ nombres dans une dimension $n$ (2 dans le plan, 3 dans l'espace...)

\begin{defi}[Vecteur]
  Un vecteur est caractérisé par :
  \begin{itemize}
    \item Une longueur
    \item Un sens
    \item Une direction
  \end{itemize}
\end{defi}

Le vecteur de longueur 0 est unique et noté $\overrightarrow{0}$

\begin{form}[Addition]
Dans le plan, deux vecteurs $\overrightarrow{u} (x,y)$, $\overrightarrow{v} (x',y')$ ont pour somme :\\

\centering{$\overrightarrow{u} + \overrightarrow{v} = (x+x',y+y')$}
\end{form}
Le principe pour la soustraction reste le même.

\begin{form}[Multiplication]
  Dans le plan, $\overrightarrow{u} (x,y)$ et $\alpha \in \mathbb{R}$, alors
  \centering{$\alpha\overrightarrow{u} = (\alpha x, \alpha y)$}
\end{form}

\begin{defi}[Coolinearité]
  Deux vecteurs sont colineraires si ils ont la même direction, c'est à dire qu'il existe $\alpha \in \mathbb{R}$ tel que
  $\overrightarrow{u} = \alpha \overrightarrow{v}$
\end{defi}

\begin{defi}[Produit Scalaire]
  Comme son nom le dit si bien, le produit scalaire renvoie un scalaire (grosso merdo un nombre) et pas un vecteur.\\
  Dans le plan $\overrightarrow{u} (x,y)$, $\overrightarrow{v} (x',y')$, alors\\

  \begin{itemize}
    \item $\overrightarrow{u} . \overrightarrow{v} = xx' + yy'$ ; où `.` dénote du produit scalaire
    \item $\overrightarrow{u} . \overrightarrow{v} = |\overrightarrow{u}| |\overrightarrow{v}| \cos{\Theta}$  ; où $\Theta$
      dénote de l'angle formé par les 2 vecteurs.
  \end{itemize}
\end{defi}

\subsection{Forme trigonométrique}
\begin{defi}[Forme Trigonométrique d'un nombre complexe]
Un point M dans le plan est donc représentable soit par son abscisse/ordonnée, soit par la longueur du vecteur $\overrightarrow{OM}~note~r$ ainsi que l'angle que forme ce vecteur avec la droite des abscisses noté $\Theta$\\
Avec $z = a + ib$, on a donc $z = r(cos\Theta + isin\Theta)$\\
Cette forme est appellée, forme trigonométrique de $z$, où $r = |z|$, et $\Theta$ est appellé argument de $z$ ; $arg(z) = \Theta + 2k\pi,~k \in \mathbb{Z}$
\end{defi}
Que signifie ce $2k\pi$, nous savons que $2\pi$ radian = $360°$, donc que si nous tournons de $2\pi$ en radian, nous allons faire un tour complet. Or, un truc complet, signifie donc que nous sommes revenue à notre position initiale. Notons $a = n(cos\Theta + isin\Theta)$, rajoutons $2\pi$ à $\Theta$. Le fait que cos et sin soient cyclique sur $2\pi$, fait donc que $\Theta\equiv\Theta+2k\pi [2\pi]$. Lisez 'theta est congru à theta + 2 k pi modulo 2 pi'.

\subsection{Forme exponentielle}
\begin{defi}[Forme exponentielle d'une nombre complexe]
  Connaissant le module et l'argument de z, nous pouvons écrire que :\\

  \centering{$z = re^{i\Theta}$}
\end{defi}
Propriétes importantes :
\begin{itemize}
    \item Formule d'Euler : $$\forall \Theta \in \mathbb{R},~cos\Theta~=~\frac{e^{i\Theta}+e^{-i\Theta}}{2}~et~sin\Theta~=~\frac{e^{i\Theta}-e^{-i\Theta}}{2i}$$
    \item Formule de Moivre : $(cos\Theta+isin\Theta)^{n} = cos(n\Theta) + isin(n\Theta)$
\end{itemize}
On peut ainsi linéariser $\cos^{n}\Theta$ et $\sin^{n}\Theta$, voici pour $\cos$:

\begin{center}
  $$Avec~la~formule~d'Euler~:\cos^{n}\Theta$$
  $$=~(\frac{e^{i\Theta}+e^{-i\Theta}}{2})^{n}$$
  $$= \frac{1}{2^{n}}(e^{i\Theta}+e^{-i\Theta})^{n}$$
  $$Avec~le~binome~de~Newton~:~(x+y)^{n} = \sum_{k=0}^{n}{\binom{n}{k}x^{k}y^{n-k}} = \sum_{k=0}^{n}{\binom{n}{k}x^{n-k}y^{k}}$$
  $$= \frac{1}{2^{n}}\sum_{k=0}^{n}{\binom{n}{k}(e^{i\Theta})^{(n-k)}(e^{-i\Theta})^{k}}$$
  $$= \frac{1}{2^{n}}\sum_{k=0}^{n}{{\binom{n}{k}e^{i\Theta(n)-i\Theta(k)}e^{-i\Theta(k)}}}$$
  $$= \frac{1}{2^{n}}\sum_{k=0}^{n}{\binom{n}{k}e^{i\Theta(n-2k)}}$$
\end{center}

\subsection{Minis-exos 2}
\begin{itemize}
  \item Les vecteurs $\overrightarrow{u}(7,14),~\overrightarrow{v}(1.4, 2.8)$ sont-ils colineraires ?\\
    Donnez leur produit scalaire, leur addition ainsi que leur soustraction
  \item Donnez la forme trigonométrique et exponentielle de : $7 + 8i$, $5 - 4i$, $\pi - e^{7i}$
  \item Linearisez $sin^{n}\Theta$
\end{itemize}

\section{Racine n-ème}

\subsection{Racine n-ème signification et comment les trouver}
\begin{defi}
  Les racines n-ème d'un nombre complexe $a + ib$ sont tous nombres complexes $z$ vérifiant :
  \begin{center}
    $z^{n} = a + ib$
  \end{center}
\end{defi}
Ces racines n-ème peuvent être trouvés à l'aide de cette (trop grosse) formule:
\[z_{k} = \sqrt[n]{r}(\cos(\frac{\Theta}{n} + \frac{2k\pi}{n}) + i\sin(\frac{\Theta}{n} + \frac{2k\pi}{n})) = \sqrt[n]{r}e^{i(\frac{\Theta}{n} + \frac{2k\pi}{n})}\] avec $k \in\{x \in\mathbb{N}~|~x~\leq~n-1\}$, $\Theta=\arg (a + ib)$, $r = |a + ib|$\\
Exemple d'utilisation:
Posons l'équations $z^{3} = a + ib$, les solutions seront donc:
\begin{itemize}
  \item $\sqrt[3]{|a + ib|}e^{i (\frac{\arg (a + ib)}{3} +\frac{2*0\pi}{3})}$
  \item $\sqrt[3]{|a + ib|}e^{i (\frac{\arg (a + ib)}{3} +\frac{2*1\pi}{3})}$
  \item $\sqrt[3]{|a + ib|}e^{i (\frac{\arg (a + ib)}{3} +\frac{2*2\pi}{3})}$
\end{itemize}

\subsection{Racine n-ème de L'unité}
Les racines n-ème de l'unité sont tous $z$ vérifiant $z^{n} = 1$\\
Pour résoudre les racines n-ème de l'unité, posons $\omega_{n} = e^{i\frac{2\pi}{n}}$, les solutions seront les $k$ puissances de $\omega_{n}$, avec $k \in \{x \in\mathbb{N}~|~x~\leq~n-1\}$\\
Posons l'équations $z^{3} = 1$, les solutions seront donc:
\begin{itemize}
  \item $\omega_{3}^{0}=e^{\frac{i2\pi*0}{3}}$
  \item $\omega_{3}^{1}=e^{\frac{i2\pi*1}{3}} $
  \item $\omega_{3}^{2}=e^{\frac{i2\pi*2}{3}}$
\end{itemize}

\subsection{Minis Exos 3}
\begin{itemize}
  \item Trouvez les racines n-ème de $z^{7} = 7 + 7i$
  \item trouvez les racinzs n-ème de l'unité de $z^{10}$
\end{itemize}

\section{Géometrie dans le plan complexe}

Bon je suppose que la géometrie basique dans le plan complexe est assez trivial, et surtout déjà vu en terminale, pour faire une partie du cours dessus. je vais donc direct commencé par les transformations.

\subsection{Transformations}
Un transformation d'un plan $\mathbb{P}$ est une bijection dans lui-même. Ainsi, il associe chacun des ses points, noté $a$ à un autre unique points $b$, tel que $a,b \in \mathbb{P}^{2}$. Notons $T$ cette transformation.\\
Le fait que ce soit une bijection, on garde ses propriétés. Comme pour une application, la transformation réciproque est notée $T^{-1}$.\\
Une transformation est une similitude si elle conserve le rapport de distances. Pour simplifier, prenons pour tous duo de points $a,b$ qui sont distants de $n$ unité(s) dans le plan $\mathbb{P}$, alors ils seront séparés de $\lambda n$ unité(s) après transformation $T$ de $\mathbb{P}$ avec $\lambda > 0$.\\
On dit que le point $A$ est invariant $ssi$ $A = T(A)$

\begin{defi}
Une similitude directe est une transformation tel que, $T(z)=az+b$ avec $(a,b) \in \mathbb{C}^{*} * \mathbb{C}$
\end{defi}
TODO
\end{document}
