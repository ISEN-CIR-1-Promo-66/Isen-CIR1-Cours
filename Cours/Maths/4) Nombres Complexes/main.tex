\documentclass{article}
\usepackage[utf8]{inputenc}
\usepackage{amsfonts}
\usepackage{amssymb}
\usepackage{amsmath}
\newtheorem{defi}{Définition}
\newtheorem{form}{Formules}
\title{(4) Nombres Complexes}
\author{Pim-Simon Hauguel}
\date{13-10-2020 (Trop tard)}

\begin{document}


\maketitle

\tableofcontents

\section{Excuses}
Bon déjà, première chose : Je suis désolé de vous avoir laché durant ces trop longs 30 jours qui nous sépare de mon dernier paper. La prochaine fois, il faut me taper sur les doigts ! Bref, je vais essayer de reprendre le rythme que je m'étais imposé au début et arreter de faire du hors programme. Commencons !

\section{Qu'est-ce qu'un nombre premier ?}
\begin{defi}[Définition et écriture]
  L'ensemble des nombres complexes est noté $\mathbb{C}$, on dit que
  $\mathbb{R} \subset \mathbb{C}$.
  L'unité dite imaginaire, est noté $i$ et est définie comme : $i = \sqrt{-1}$, donc que $i^{2} = -1$.
  La forme $z = a + ib$ est appelée \textit{notation cartésienne} de z (z est complexe).
  Avec $a$ la partie réelle ($Re(z)$), et $b$ la partie imaginaire ($Im(z)$). ($a,b \in \mathbb{R}^{2}$)
\end{defi}
\subsection{Ce que l'on en déduit}
Avec ces informations, certaines informations peuvent être remarquées :\\
\begin{itemize}
  \item $\forall z \in \mathbb{C},\exists! (a,b) \in \mathbb{R}^{2}(z = a + ib)$\\
    Pour le dire d'une manière plus accesible, un nombre complexe $z$ a une unique écriture cartésienne.
  \item $i\mathbb{R}$ est l'ensemble des imaginaires purs. Des exemples d'imaginaires purs ?
    \begin{itemize}
      \item $0 + 7i = 7i$
      \item $0 + i = i$
      \item $i^{3} = i^{2} * i = -1 * i = -i)$
    \end{itemize}
  \item Il existe aussi des réels purs, le principe est le même, simplement c'est la partie imaginaire qui est nulle.
\end{itemize}
\subsubsection{Le conjugué}
\begin{form}[Formule stricte]
  Le conjugué d'un nombre complexe $z$ est défini ainsi :

\centering{$conjugue(z) = \overline{z} = Re(z) - Im(z)$}
\end{form}
Pour résumé, le conjugué d'un nombre complexe est simplement ce même nombre, mais en opposant sa partie imaginaire.\\
On peut donc en déduire cela : $\overline{\overline{z}} = z$
\subsubsection{Minis Exos 1}
En connaissant ces informations, veuillez généraliser ces formules (trouver le $?$):
\begin{itemize}
  \item $\forall z \in \mathbb{C}(z*\overline{z} = ?)$
  \item $\forall z \in \mathbb{C}(z + \overline{z} = ? \in ?)$
  \item $\forall z \in \mathbb{C}(z \in i\mathbb{R} \Leftrightarrow \overline{z} = ?$
\end{itemize}
(bon ils étaient triviaux, mais bon...)

\section{Représentation des nombres complexes}
On peut exprimer les nombres complexes comme des points dans un plan.\\
Posons $z = a + ib$, $a$ va représenter l'abscisse, $b$ l'ordonnée. (Il est très facile de se visualiser cela dans le plan réel, cependant, nous sommes bien ici dans le plan complexe)\\
Par exemple, le nombre $8 + 7i$ sera de coordonnées $(8,7)$\\$-4 + 0.5i$ en $(-4,0.5)$\\

Posons $M$, un point du plan complexe de centre O,  de coordonnées $(a,b)$, quel sera sa longueur ? Il est évident que l'on répondra formule de pythagore, avec du coup $OM = \sqrt{a^{2} + b^{2}}$. Et bien je suis d'accord ! Mais avec les nombres complexes, on appelera cette longueur le module de $a + ib$.\\
\begin{form}{Module}
$\centering{\forall z \in \mathbb{C}(|z| = \sqrt{z\overline{z}} = \sqrt{a^{2} + b^{2}})~with~z~=~a~+~ib}$
\end{form}

\subsection{Les Vecteurs}
Les grandeurs dites scalaires, sont caractérisées par un seul nombre, a contrario des grandeurs dites vectorielles qui sont caractérisées par $n$ nombres dans une dimension $n$ (2 dans le plan, 3 dans l'espace...)

\begin{defi}[Vecteur]
  Un vecteur est caractérisé par :
  \begin{itemize}
    \item Une longueur
    \item Un sens
    \item Une direction
  \end{itemize}
\end{defi}

Le vecteur de longueur 0 est unique et noté $\overrightarrow{0}$

\begin{form}[Addition]
Dans le plan, deux vecteurs $\overrightarrow{u} (x,y)$, $\overrightarrow{v} (x',y')$ ont pour somme :\\

\centering{$\overrightarrow{u} + \overrightarrow{v} = (x+x',y+y')$}
\end{form}
Le principe pour la soustraction reste le même.

\begin{form}[Multiplication]
  Dans le plan, $\overrightarrow{u} (x,y)$ et $\alpha \in \mathbb{R}$, alors
  \centering{$\alpha\overrightarrow{u} = (\alpha x, \alpha y)$}
\end{form}

\begin{defi}[Coolinearité]
  Deux vecteurs sont colineraires si ils ont la même direction, c'est à dire qu'il existe $\alpha \in \mathbb{R}$ tel que
  $\overrightarrow{u} = \alpha \overrightarrow{v}$
\end{defi}

\begin{defi}[Produit Scalaire]
  Comme son nom le dit si bien, le produit scalaire renvoie un scalaire (grosso merdo un nombre) et pas un vecteur.\\
  Dans le plan $\overrightarrow{u} (x,y)$, $\overrightarrow{v} (x',y')$, alors\\

  \begin{itemize}
    \item $\overrightarrow{u} . \overrightarrow{v} = xx' + yy'$ ; où `.` dénote du produit scalaire
    \item $\overrightarrow{u} . \overrightarrow{v} = |\overrightarrow{u}| |\overrightarrow{v}| \cos{\Theta}$  ; où $\Theta$
      dénote de l'angle formé par les 2 vecteurs.
  \end{itemize}
\end{defi}

\subsection{Forme trigonométrique}
TODO








\section{Solution minis-exos}
\subsection{Minis Exos 1}
\begin{itemize}
  \item $\forall z \in \mathbb{C}(z*\overline{z} = a^{2}+b^{2})$
  \item $\forall z \in \mathbb{C}(z + \overline{z} = 2a \in \mathbb{C})$
  \item $\forall z \in \mathbb{C}(z \in i\mathbb{R} \Leftrightarrow \overline{z} = -z)$
\end{itemize}
\end{document}
