\documentclass{article}
\usepackage[utf8]{inputenc}
\usepackage{amsfonts}
\title{4) les nombres complexes}
\author{Pim - Simon Hauguel}
\date{16-09-2020}
\usepackage[utf8]{inputenc}
\newtheorem{theorem}{Theoreme}

\begin{document}

\maketitle

\section{Généralités}

\subsection{Vocabulaire, Définition et Notations}

Un nombre complexe est noté sous la forme $a + ib$ avec $a,b \in \mathbb{R}^{2}$\\ et i tel que i² = -1\\
L'ensemble des complexes est noté $\mathbb{C}$\\
Sous la forme $z = a + ib$
\begin{itemize}
  \item $a + ib$ est appelée forme algébrique de z.
  \item $a$ est appelé partie réele de z, aussi noté $Re(z)$
  \item $b$ est appelé partie imaginaire de z, aussi noté $Im(z)$
  \item Dans un plan orthonormé (O,$\vec{u}$,$\vec{v}$), le point M et le vecteur $\vec{OM}$ de coordonnées $(a,b)$ sont représentés par le complexe $a + ib$, appelé $affixe$ de M et de $\vec{OM}$
\end{itemize}

\begin{theorem}
  Deux complexes $((a + ib)~et~(a' + ib'))$ sont égaux ssi $a = a' \land b = b'$, on a donc : $a + ib = a' + ib' \Leftrightarrow a = a' et b = b'$
\end{theorem}
\begin{theorem}
  Avec 2 complexes u et v, $u + v = (a + a') + i(b + b')$ et $uv = (aa' - bb') + i(ab' + a'b)$
\end{theorem}
\end{document}
